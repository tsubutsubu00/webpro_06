\documentclass[uplatex,dvipdfmx]{jlreq} % 日本語用の標準クラス
\usepackage[fleqn,tbtags]{mathtools}    % 数式の左寄せ・配置調整
\usepackage{jlreq-deluxe}               % 多書体(太字等)の利用
\usepackage[noalphabet]{pxchfon}        % 日本語フォントの埋め込み
\usepackage{stix2}                      % 欧文書体と数式の指定
% \usepackage{hira-stix}                % ヒラギノとSTIXの併用

\usepackage{url}
\usepackage{graphicx}
\usepackage{bmpsize}
\usepackage{float}
\usepackage{natbib}
\usepackage{listings}
\usepackage{ascmac}
\usepackage{xcolor}
\usepackage[hidelinks]{hyperref}
\usepackage{tabularx}

\lstset{
  basicstyle=\ttfamily\small, % フォントとサイズ
  breaklines=true,            % 行を折り返す
  columns=fullflexible,       % 幅に合わせて調整
  frame=single,               % フレームを付ける
  backgroundcolor=\color{lightgray!10}, % 背景色
  keywordstyle=\color{blue}\bfseries, % キーワードの色と太字
  commentstyle=\color{green!50!black}, % コメントの色
  stringstyle=\color{red},    % 文字列の色
  numbers=left,               % 行番号を左に表示
  numberstyle=\tiny\color{gray}, % 行番号のスタイル
  showstringspaces=false,     % 文字列中のスペースを表示しない
  tabsize=2,                  % タブサイズ
  captionpos=b,               % キャプションの位置 (b=bottom)
  language=JavaScript,        % 言語設定
}

\begin{document}
\title{Webプログラミング レポート課題}
\author{25G1087 粒良梁雅}
\maketitle

\section{githubのリポジトリのURL}
\url{https://github.com/tsubutsubu00/webpro_06.git}
\section{開発者向けマニュアル}
\subsection{1つ目:はま寿司メニュー一覧}
\subsubsection{概要}
本webアプリケーションは,現在はま寿司で取り扱っているメニューを効率的に管理・閲覧するためのシステムである.
このあと紹介する2つ目,3つ目と異なり,データ量が多いため,情報をカテゴリごとに分類し,その後メニュー一覧に
飛ぶようにしている点が特徴である.また,メニュー一覧にそれぞれ追加,編集,削除が行える機能を追加した.
変更点として,本システムはパスパラメータの数値を直接参照するのではなく,各データが持つ固有の\texttt{id}
を\texttt{find}メソッドを使って検索し,合致するオブジェクトを特定するロジックに変更した.
このようなロジックに変更した理由として,はま寿司のシステムは,メニューの追加や削除が頻繁に行われる可能性が高いという
観点から不適切であると判断したためである.具体的には,メニューを削除をした後に追加を行うと,
それ以降のデータの順番がずれてしまうが,データに紐づいているIDは変化しないためである.

\subsubsection{データ構造}
はま寿司カテゴリ一覧のデータ構造を表\ref{hyou:hama_data_kouzou_kategori}に,
はま寿司メニュー一覧のデータ構造を表\ref{hyou:hama_data_kouzou_menu}に示す.
本webアプリケーションでは,すべてのメニューデータを\texttt{hama\_all\_menu}
に格納している.この\texttt{hama\_all\_menu}は,カテゴリ名である\texttt{limited\_menu}
や\texttt{nigiri}などを持っているオブジェクトであり,値として各メニューの配列を保持する多階層構造
となっている.このような設計になっているため,サーバー側で\texttt{req.params.url}を変数として利用して,
動的にデータを切り替えて取得することが可能である.
また,項目内の\texttt{id}は,number型で定義している.この\texttt{id}を活用して,詳細表示や編集,削除
を行う際に特定のデータを識別するためのパスパラメータとして利用している.

\begin{table}[H]
\centering
\caption{はま寿司カテゴリ一覧のデータ構造}
\begin{tabularx}{\textwidth}{c|c|X|c}
    \hline
    項目名 & データ型 & 説明 & 出力例 \\
    \hline\hline
    id & number & 各カテゴリに割り当てたID & 0,1,2など \\
    \hline
    url & string & パラメータ名 & limited\_menu.nigiriなど \\
    \hline
    tag & string & カテゴリの名称 & 期間限定,にぎりなど \\
    \hline
\end{tabularx}
\label{hyou:hama_data_kouzou_kategori}
\end{table}

\begin{table}[H]
\centering
\caption{はま寿司メニュー一覧のデータ構造}
\begin{tabularx}{\textwidth}{c|c|X|c}
    \hline
    項目名 & データ型 & 説明 & 出力例 \\
    \hline\hline
    id & number & 各メニューに割り当てたID & 0,1,2など \\
    \hline
    name & string & メニューの名称 & 厳選まぐろ中とろなど \\
    \hline
    price & string & メニューの価格 & 110円(税込),176円(税込)など \\
    \hline
    suuryou & string & メニューの数量 & 1貫,2貫など \\
    \hline
    omochikaeri & string & そのメニューがお持ち帰りできるかの可否 & お持ち帰り可など \\
    \hline
\end{tabularx}
\label{hyou:hama_data_kouzou_menu}
\end{table}

\subsubsection{ページ遷移}
%このページ遷移の中にHTTPメソッドとリソース名を入れる
%ページ遷移図も
\label{page_senni:hama}
本webアプリケーションがどのようにページ遷移をするかを表\ref{hyou:hama_page_senni}に,
それらを簡潔にまとめた図を図\ref{sennizu:hama}に示す.
本webアプリケーションでは,まず\texttt{/menu}にてカテゴリ一覧を表示し,カテゴリ選択をすると
指定されたメニュー一覧\texttt{/menu/:url}にページ遷移する仕様となっている.
また,本システムは追加,編集,削除のCRUD操作にも対応しており,それらの操作を行うと,ユーザーが指定していた
カテゴリのメニュー一覧へとリダイレクト処理が行われる設計となっている.

\begin{table}[H]
\centering
\caption{はま寿司メニュー一覧のページ遷移}
\begin{tabularx}{\textwidth}{c|X|c|X}
    \hline
    目的 & リソース名 & HTTPメソッド & 遷移先 \\
    \hline\hline
    カテゴリ一覧 & /menu & GET & hama.ejs \\
    \hline
    メニュー一覧 & /menu/:url & GET & hama\_menu.ejs \\
    \hline
    追加フォーム & /menu/:url/create & GET & /public/hama\_menu\_new.html \\
    \hline
    新規追加 & /menu/:url/create & POST & /menu/:url (メニュー一覧に戻る) \\
    \hline
    詳細表示 & /menu/:url/:number & GET & hama\_menu\_detail.ejs \\
    \hline
    編集 & /menu/:url/edit/:number & GET & hama\_menu\_edit.ejs \\
    \hline
    更新 & /menu/:url/update/:number & POST & /menu/:url (メニュー一覧に戻る) \\
    \hline
    削除 & /menu/:url/delete/:number & GET & /menu/:url (メニュー一覧に戻る) \\
    \hline
\end{tabularx}
\label{hyou:hama_page_senni}
\end{table}

\begin{figure}[H]
\centering
\includegraphics[width=12cm]{figs/hama_flow.png}
\caption{はま寿司メニュー一覧のページ遷移図}
\label{sennizu:hama}
\end{figure}

\subsubsection{リソースごとの機能の詳細}
第\ref{page_senni:hama}節で作成した表を基に,リソースごとの機能の詳細について説明する.
\texttt{/menu}のカテゴリを一覧表示するリソースでは,すべてのカテゴリ情報を持つ\texttt{hama\_menu}
を\texttt{hama.ejs}に渡し,ユーザーに見たいページの選択を提示する役割を持つ.ユーザーに選択された
カテゴリは,その後,メニュー一覧へと遷移する.
\texttt{/menu/:url}のメニューを一覧表示するリソースでは,パスパラメータ\texttt{url}
をキーとして利用し,連想配列である\texttt{hama\_all\_menu}から該当するカテゴリの配列を動的に抽出した後に,
表示する役割を持っている.ユーザーが選んだメニューを押すと,詳細表示へと遷移する.
\texttt{/menu/:url/:number}の詳細表示を行うリソースでは,\texttt{:number}を固有のIDとして扱う.
\texttt{find}メソッドを用いて,\texttt{item.id == number}により検索をかけることで,
配列の順序に依存しないようなデータ抽出を行うことができる.
\texttt{/menu/:url/create}の新規追加を行うリソースでは,\texttt{res.sendFile}を用いて,
\texttt{hama\_new.html}を読み込む.フォーム送信時には,既存の最終データのIDに1を加算する
\texttt{new\_id}を生成し,新しく作成したデータを配列の末尾に\texttt{push}により
追加する役割を持っている.
\texttt{/menu/:url/edit/:number}の編集を行うリソースでは,詳細表示でも利用した
\texttt{find}メソッドで特定したオブジェクトの各要素を,編集画面で表示し編集する役割を持っている.
\texttt{/menu/:url/update/:number}の更新を行うリソースでは,詳細表示や編集と同様
\texttt{find}メソッドで特定したオブジェクトの各要素を,フォームから送信されたデータで直接上書き
する機能を持っている.
\texttt{/menu/:url/delete/:number}の削除を行うリソースでは,\texttt{:number}によって
パスパラメータで指定された\texttt{ID}と合致するデータを\texttt{find}メソッドにより特定した後,
\texttt{splice}メソッドによって配列から削除する役割を持っている.また,誤って削除ボタンを押した際に
\texttt{onclick="return confirm()}を導入することで,誤って削除することを防ぐ利用者への配慮も行った.

\subsection{2つ目:tex数学記号一覧}
\subsubsection{概要}
本webアプリケーションは,LaTexで使用される数学記号を一覧表示するシステムである.
閲覧するだけでなく,数学記号をブラウザ上で追加・編集・削除のできる機能を備えている.

\subsubsection{データ構造}
\label{data_kouzou:tex}
tex数学記号一覧のデータ構造を表\ref{hyou:tex_data_kouzou}に示す.
項目名に示しているこれらのデータは,すべて\texttt{tex\_data}に格納している.
\texttt{id}は,number型で定義しており,それ以外の要素は,文字列として出力するために,
string型で定義されている.特に,\texttt{id}の数値を活用して,詳細表示や編集,削除を行う際に
特定のデータを識別するためのパスパラメータとして利用している.

\begin{table}[H]
\centering
\caption{tex数学記号一覧のデータ構造}
\begin{tabularx}{\textwidth}{c|c|X|X}
    \hline
    項目名 & データ型 & 説明 & 出力例 \\
    \hline\hline
    id & number & 数学記号に割り当てたID & 0,1,2など \\
    \hline
    symbol & string & texで出力される数学記号 & =,≠など \\
    \hline
    name & string & 数学記号の名称 & 等号,不等号など \\
    \hline
    command & string & texで使うコマンド & =,\textbackslash{}neqなど \\
    \hline
    genre & string & 数学記号のジャンル & 等号,不等号,演算子など \\
    \hline
    mean & string & 数学記号の持つ意味 & 等しいことを示す,等しくないことを示す,など \\
    \hline
\end{tabularx}
\label{hyou:tex_data_kouzou}
\end{table}
    
\subsubsection{ページ遷移}
%このページ遷移の中にHTTPメソッドとリソース名を入れる
%ページ遷移図も
\label{page_senni:tex}

本webアプリケーションがどのようにページ遷移をするかを表\ref{hyou:tex_page_senni}に,
それらを簡潔にまとめた図を図\ref{sennizu:tex}に示す.
本システムのトップページは,\texttt{\textbackslash{}tex}で,「一覧表示」の役割を担っている.
CRUDの要素である追加,編集,削除を行うと,遷移先としてトップページである\texttt{\textbackslash{}tex}
にリダイレクト処理によって戻るような設計となっている.このCRUDによって,ユーザーインターフェース
向上につなげている.

\begin{table}[H]
\centering
\caption{tex数式記号一覧のページ遷移}
\begin{tabularx}{\textwidth}{c|X|c|X}
    \hline
    目的 & リソース名 & HTTPメソッド & 遷移先 \\
    \hline\hline
    一覧表示 & /tex & GET & tex.ejs \\
    \hline
    追加フォーム & /tex/create & GET & /public/tex\_new.html \\
    \hline
    詳細表示 & /tex/:number & GET & tex\_detail.ejs \\
    \hline
    追加 & /tex & POST & /tex (一覧表示に戻る) \\
    \hline
    編集 & /tex/edit/:number & GET & tex\_edit.ejs \\
    \hline
    更新 & /tex/update/:number & POST & /tex (一覧表示に戻る) \\
    \hline
    削除 & /tex/delete/:number & GET/POST & /tex (一覧表示に戻る) \\
    \hline
\end{tabularx}
\label{hyou:tex_page_senni}
\end{table}

\begin{figure}[H]
\centering
\includegraphics[width=12cm]{figs/tex_flow.png}
\caption{tex数式記号一覧のページ遷移図}
\label{sennizu:tex}
\end{figure}

\subsubsection{リソースごとの機能の詳細}
第\ref{page_senni:tex}節で作成した表を参考に,リソースごとの各機能について説明する.
一覧表示することが目的である\texttt{/tex}のリソースでは,配列\texttt{tex\_data}の中に
格納されているデータをすべて取得し,\texttt{tex.ejs}に渡して表形式で表示する役割を持っている.
その\texttt{/tex}に表示されるデータはURLとして\texttt{/tex/:number}の詳細表示のリンクへ
とべるようになっている.
\texttt{/tex/:number}のリソースでは,URLから取得したパスパラメータ\texttt{:number}
を配列の番号として割り当て,\texttt{tex\_data[number]}によって特定のデータを抽出して
\texttt{tex\_detail.ejs}を使って表示する役割を持っている.
\texttt{/tex/create}や\texttt{/tex}の追加処理を行うリソースでは,追加ボタンを押すと,
\texttt{/tex/create}にアクセスし,\texttt{tex\_new.html}にリダイレクトし,入力画面を表示する.
フォーム送信時には,既存の最終データのIDに1を加算する\texttt{new\_id}を生成し,新しく作成した
データを配列の末尾に\texttt{push}により追加する役割を持っている.
\texttt{/tex/edit/:number}の編集を行うリソースでは,\texttt{tex\_data[number]}から
現在の値を読み込み,それを編集画面で表示する役割を持っている.
\texttt{/tex/update/:number}の更新を行うリソースでは,送信されたデータで配列の該当する部分
\texttt{tex\_data[number]}の各要素を直接上書きする機能を持っている.
\texttt{/tex/delete/:number}の削除を行うリソースでは,パスパラメータで指定した場所を
\texttt{splice}というコードの第1引数に渡し,配列から該当する要素を1件削除する.削除
を行ったあとは,tex数学記号一覧のページである\texttt{/tex}にリダイレクトする.
また,誤って削除ボタンを押した際に\texttt{onclick="return confirm()}を導入することで,
誤って削除することを防ぐ利用者への配慮も行った.

\subsection{3つ目:大乱闘スマッシュブラザーズファイター一覧}
\subsubsection{概要}
本webアプリケーションは,nintendo Switch用ゲームソフト
「大乱闘スマッシュブラザーズ」(以降,スマブラ)に登場するファイター(キャラクター)
全員を一覧表示するシステムである.一覧表示だけでなく,ファイター
の追加や編集,削除を行うことができる.

\subsubsection{データ構造}
スマブラファイター一覧のデータ構造を表\ref{hyou:sumabura_data_kouzou}に示す.
項目名として示しているデータは,すべて\texttt{sumabura\_data}に格納している.
第\ref{data_kouzou:tex}節と同様に,\texttt{id}は,number型で定義しており,
それ以外の要素は,文字列として出力するために,string型で定義している.特に,\texttt{id}
の数値を活用して,詳細表示や編集,削除を行う際に特定のデータを識別するためのパスパラメータとして利用している.

\begin{table}[H]
\centering
\caption{スマブラファイター一覧のデータ構造}
\begin{tabularx}{\textwidth}{c|c|X|X}
    \hline
    項目名 & データ型 & 説明 & 出力例 \\
    \hline\hline
    id & number & 各ファイターに割り当てたID & 0,1,2など \\
    \hline
    name & string & ファイター名 & マリオ,ドンキーコングなど \\
    \hline
    series & string & 各ファイターが登場する作品 & スーパーマリオ,ドンキーコングなど \\
    \hline
    nannido & string & プレイヤーの上位5\%であるVIPに到達する難易度 & ★★★★☆など \\
    \hline
\end{tabularx}
\label{hyou:sumabura_data_kouzou}
\end{table}

\subsubsection{ページ遷移}
%このページ遷移の中にHTTPメソッドとリソース名を入れる
%ページ遷移図も
\label{page_senni:sumabura}
本webアプリケーションがどのようにページ遷移をするかを表\ref{hyou:sumabura_page_senni}に,
そのページ遷移図を図\ref{sennizu:sumabura}に示す.本システムのトップページは,\texttt{/sumabura}
で,「一覧表示」の役割を担っている.第\ref{page_senni:tex}節と同様に,CRUDの要素である追加,編集,削除を行うと,
遷移先としてトップページである\texttt{/sumabuta}にリダイレクト処理によって戻るような設計となっている.
このCRUDによって,ユーザーインターフェース向上につなげている.

\begin{table}[H]
\centering
\caption{スマブラファイター一覧のページ遷移}
\begin{tabularx}{\textwidth}{c|X|c|X}
    \hline
    目的 & リソース名 & HTTPメソッド & 遷移先 \\
    \hline\hline
    一覧表示 & /sumabura & GET & sumabura.ejs \\
    \hline
    追加フォーム & /sumabura/create & GET & /public/sumabura\_new.html \\
    \hline
    詳細表示 & /sumabura/:number & GET & sumabura\_detail.ejs \\
    \hline
    追加 & /sumabura & POST & /sumabura (一覧表示に戻る) \\
    \hline
    編集 & /sumabura/edit/:number & GET & sumabura\_edit.ejs \\
    \hline
    更新 & /sumabura/update/:number & POST & /sumabura (一覧表示に戻る) \\
    \hline
    削除 & /sumabura/delete/:number & GET & /sumabura (一覧表示に戻る) \\
    \hline
\end{tabularx}
\label{hyou:sumabura_page_senni}
\end{table}

\begin{figure}[H]
\centering
\includegraphics[width=12cm]{figs/sumabura_flow.png}
\caption{スマブラファイター一覧のページ遷移図}
\label{sennizu:sumabura}
\end{figure}

\subsubsection{リソースごとの機能の詳細}
第\ref{page_senni:sumabura}節で作成した表を参考に,リソースごとの各機能について説明する.
\texttt{/sumabura}の一覧表示するリソースでは,配列\texttt{sumabura\_data}に格納している
データを\texttt{sumabura.ejs}に渡して表形式で表示する役割を持っている.その\texttt{/sumabura}
に表示されるデータはURLとして機能し,\texttt{/sumabura/:number}の詳細表示のリンクとして
とべるようになっている.\texttt{/sumabura/:number}の詳細表示を行うリソースでは,URLから
取得したパスパラメータ\texttt{:number}を配列の番号として割り当て,\texttt{sumabura\_data[number]}
によって特定のデータを抽出して\texttt{sumabura\_detail.ejs}にで表示する役割を持っている.
\texttt{/sumabura/create}や\texttt{/sumabura}の追加の処理を行うリソースでは,追加ボタン
を押すと,\texttt{/sumabura/create}にアクセスし,\texttt{sumabura\_new.html}にリダイレクトすることで,
入力画面を表示する.フォーム送信時には,既存の最終データのIDに1を加算する\texttt{new\_id}を生成し,
新しく作成したデータを配列の末尾に\texttt{push}により追加する役割を持っている.\texttt{/sumabura/edit/:sumabura}
の編集を行うリソースでは,\texttt{sumabura\_data[number]}から,現在の値を読み込み,
それを編集画面で表示する役割を持っている.\texttt{/sumabura/update/:number}の更新を行うリソースでは,
送信されたデータで配列の該当する部分\texttt{sumabura\_data[number]}の各要素を直接上書きする機能を持っている.
\texttt{/sumabura/delete/:number}の削除を行うリソースでは,パスパラメータで指定した場所を\texttt{splice}
というコードの第1引数に渡し,配列から該当する要素を1件削除する.削除を行ったあとは,ファイター一覧のページである
\texttt{/sumabura}にリダイレクトする.また,誤って削除ボタンを押した際に\texttt{onclick="return confirm()}
を導入することで,誤って削除することを防ぐ利用者への配慮も行った.

\section{管理者向けマニュアル}
\subsection{インストール方法}
本webアプリケーションを利用するには,JavaScript実行環境を整える必要がある.
具体的には,\texttt{node.js}をダウンロードする必要がある.
\texttt{node.js}はJavaScriptでOSの機能にアクセスするプログラムを書くことができるため,
導入が必須である.\par
そのような\texttt{node.js}をダウンロードするには,ターミナルを開き,ルートディレクトリで以下のコマンドを実行する.
\begin{figure}[H]
    \begin{lstlisting}[caption=node.jsのダウンロード方法, label=node.js_download, escapechar=\@]
nodebrew install stable
npm install -g npm
    \end{lstlisting}
\end{figure}

次に,\texttt{node.js}がダウンロードできているか確認するために,以下のコードを実行する.
\begin{figure}[H]
    \begin{lstlisting}
node -v
    \end{lstlisting}
\end{figure}
このコードを実行したら,\texttt{v24.3.0}などのように表示されれば,\texttt{node.js}
のダウンロードが完了し,実行環境の構築完了である.

\subsection{起動方法}
ターミナルを開き,対象ディレクトリ(ここでは\texttt{webpro\_06})まで移動し,
以下のコードを実行することでWebサーバーを立ち上げることができる.
\begin{figure}[H]
    \begin{lstlisting}[caption=Webサーバーの起動方法, label=server_begin, escapechar=\@]
node app_kadai.js
    \end{lstlisting}
\end{figure}
起動に成功すると,ターミナル上に\texttt{Example app listening on port 8080!}
というメッセージが出力される.

\subsection{起動できない場合の対処法}
起動ができない場合の主な原因は2つある.第一に,ポートを複数開いている可能性がある.
具体的には,すでに他のプログラムが8080番ポートを使用している場合,起動に失敗する.
このような場合は,使っていない方のプログラムのポートを終了させる必要がある.
第二に,ディレクトリ指定を間違える原因である.これは,\texttt{app\_kadai.js}のある
ディレクトリではない場所でソースコード\ref{server_begin}を実行してしまうことで起動できない場合がある.
このような場合においては,\texttt{ls}コマンドで現在どの位置にいるかを確認し,正しいディレクトリへ
移動することで解決することができる.

\subsection{終了方法}
サーバーを停止させるには,ソースコード\ref{server_begin}を実行中のターミナルで以下の
コマンドを入力することで停止することができる.
\begin{figure}[H]
    \begin{lstlisting}[caption=Webサーバーの終了方法, label=server_end, escapechar=\@]
Ctrl + C
    \end{lstlisting}
\end{figure}

\subsection{わかっている不具合}
わかっている不具合として,サーバーを終了すると,追加・編集・削除を行ったにも関わらず,
初期化してしまう不具合がある.これは,データをサーバー実行時の変数内に格納していたためである.
つまり,サーバーを停止したと同時にメモリ上の変数が削除されてしまうので,初期化されてしまう.
このような不具合を解決するためには,外部データベースを活用することで解決することができる.

\section{利用者向けマニュアル}
\subsection{概要}
本webアプリケーションは,はま寿司の様々なメニューをカテゴリごとに閲覧し,
必要に応じてメニューの追加や内容の修正,削除を行うことができる管理システムである.

\subsection{基本的な操作の流れ}
%使用できる機能をかく
%データの閲覧・追加・編集・削除ができることを簡潔に示す
ユーザーは主に4つの操作を直感的に行うことができる.
その4つの操作とは,閲覧,追加,編集,削除のことである.
閲覧機能では,カテゴリを選び,メニューの詳細情報を確認することができる.
追加機能では,新しいメニューをメニュー一覧に追加する.
編集機能では,既存のメニュー名や価格,お持ち帰りの可否などの要素を編集する.
削除機能では,提供終了したメニューなどを削除したい場合に用いる.

\subsection{操作手順の詳細}
%このセクションはスクリーンショットを活用する
\subsubsection{システムの起動}
%起動画面
サーバーを管理している管理者がサーバーを起動したあと,利用者は,WebブラウザのURL欄に
\texttt{https//localhost:8080/menu}と入力することで,システムを起動することができる.

\subsubsection{データの閲覧方法}
%一覧表示・詳細表示
システムを起動すると,まず図\ref{fig:kategori}のようなカテゴリ一覧が表示される.
その後,「期間限定」や「にぎり」などの見たい項目を選択すると,図\ref{fig:menu}のような
メニュー一覧が表示される.さらに,特定のメニュー名をクリックすることで,図\ref{fig:shousai}のような
価格やお持ち帰りの可否などの情報がのっている詳細画面を見ることができる.

\begin{figure}[H]
\centering
\includegraphics[width=8cm]{figs/kategori.png}
\caption{はま寿司のカテゴリ一覧}
\label{fig:kategori}
\end{figure}

\begin{figure}[H]
\centering
\includegraphics[width=8cm]{figs/menu.png}
\caption{はま寿司のメニュー一覧}
\label{fig:menu}
\end{figure}

\begin{figure}[H]
\centering
\includegraphics[width=8cm]{figs/shousai.png}
\caption{はま寿司のメニュー詳細一覧}
\label{fig:shousai}
\end{figure}

\subsubsection{新しくデータを登録する方法}
%データ追加
メニュー一覧の画面下部にある追加ボタンをクリックすると,図\ref{fig:tsuika}のような
入力画面が表示される.すべての項目を入力し,登録ボタンをクリックすることで
新しいメニューが一覧に追加される.
\begin{figure}[H]
\centering
\includegraphics[width=8cm]{figs/tsuika.png}
\caption{はま寿司のメニュー追加の方法}
\label{fig:tsuika}
\end{figure}

\subsubsection{既存データの修正方法}
%データ編集
メニューの詳細画面にある編集ボタンをクリックすると,図\ref{fig:henshuu}のような
編集画面が表示される.内容を書き換えたあとに,送信ボタンを押すことで内容が上書きされる.
\begin{figure}[H]
\centering
\includegraphics[width=6cm]{figs/henshuu.png}
\caption{はま寿司のメニュー編集の方法}
\label{fig:henshuu}
\end{figure}

\subsubsection{データの削除方法}
%データ削除
メニュー画面の詳細画面にある削除ボタンをクリックすると,図\ref{fig:sakujo}のように
確認のメッセージが表示される.OKを選択するとその対象のデータは削除される.
\begin{figure}[H]
\centering
\includegraphics[width=12cm]{figs/sakujo.png}
\caption{はま寿司のメニュー削除の方法}
\label{fig:sakujo}
\end{figure}

\end{document}

%開発者→概要のセクションにどの部分を変更したかと変更理由記述する